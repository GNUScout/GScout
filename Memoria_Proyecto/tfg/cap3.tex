%%%%%%%%%%%%%%%%%%%%%%%%%%%%%%%%%%%%%%%%%%%%%%%%%%%%%%%%%%%%%%%%%%%%%%%%%%%%%%%
% Chapter 3: Descripci�n de la Aplicaci�n
%%%%%%%%%%%%%%%%%%%%%%%%%%%%%%%%%%%%%%%%%%%%%%%%%%%%%%%%%%%%%%%%%%%%%%%%%%%%%%%

%++++++++++++++++++++++++++++++++++++++++++++++++++++++++++++++++++++++++++++++


En el c�pitulo ~\ref{chapter:intro} se describi� brevemente la aplicaci�n. En este c�pitulo nos expandiremos y eplicaremos toda la funcionalidad de GScout.

%++++++++++++++++++++++++++++++++++++++++++++++++++++++++++++++++++++++++++++++
\section{Usuarios}
\label{3:sec1}

La aplicaci�n esta destinada para el uso de los empleados de la organizaci�n de scout Aguere 70, el inicio de sesi�n esta restringido, por tanto solo podr�n 
acceder con su cuenta de Google perteneciente al dominio ``gruposcoutaguere70.com''. En principio todos los usuarios son genericos, pero la aplicaci�n esta preparada
para clasificarlos seg�n su cargo, y que cada usuario pueda desempe�ar unas funciones u otras dependiendo del cargo asignado por el administrador.\\


[[Imagen del login]]


%++++++++++++++++++++++++++++++++++++++++++++++++++++++++++++++++++++++++++++++
\section{Gesti�n de socios}
\label{3:sec2}

A continuaci�n nombraremos las funcionalidades entorno a la gesti�n de socios.

\subsection{Creaci�n de Socios}

Los usuarios podr�n crear nuevos socios rellenando los formularios predefinidos en la aplicaci�n, donde se clasifican los datos en personales, economicos, familiares y medicos.\\

[[Imagen de los formularios]]

\subsection{Edici�n de Socios}

Una vez creados los socios se pueden editar sus datos, de una manera similar a la de creaci�n de socios, simplemente buscamos el socio y le damos a editar, en la pesta�a del tipo de informaci�n
que querramos cambiar.\\

[[Imagen de formularios de edici�n]]

\subsection{Borrado de Socios}
Tambien podremos borrar los socios, solo que estos socios no son borrados permanentemente con esta opci�n, sino que su estado pasar� a inactivo y no estara visible a todo el mundo.\\

[[Imagen borrado y estado inactivo]]

\subsection{Familiares}

Es obligado que cada socio pertenezca a una familia, la cual tendra un responsable que sera el indentificador de la familia, a la que se le pueden asignar varios socios, padres y/o tutores. En cierto modo, podremos
formar lo que seria un arbol de familia entre los socios, saber que socios conviven en la misma familia, saber quienes son los padres/tutores del socio, etc.\\

[[ Imagen arbol familia]]\\

Tambien GScout nos da la opci�n de cambiar o crear una familia nueva, por cualquier circunstancia que pueda pasar, en la que se necesite modificar las familias ya existentes.\\

[[ Imagen edicion familia]]

\subsection{Medicamentos}

Muchos de los socios podr�an estar bajo medicaci�n de algun tipo, por tanto existe un modulo en el que podemos anexar a los socios los datos de sus medicamentos, como el nombre, las pautas y las dosis correspondientes.\\

[[imagen medicamentos]]

%++++++++++++++++++++++++++++++++++++++++++++++++++++++++++++++++++++++++++++++
\section{Cambios de unidad}
\label{3:sec3}

A petici�n de la organizaci�n scout Aguere 70 se cre� una funci�n que llamada \textbf{Cambio de Unidad} que consistes en analizar todos los socios, verificar su edad, y si procede cambiarlo a la secci�n/unidad acorde a su edad.\\

[tabla de unidades por rango de edades]

%++++++++++++++++++++++++++++++++++++++++++++++++++++++++++++++++++++++++++++++
\section{Listados de informaci�n}
\label{3:sec4}

Hay una secci�n en la aplicaci�n en la que podemos listar los socios visualizando sus datos dependiendo del listado, es decir, existen un listado especifico para los datos personales y otro para los datos economicos.\\

En estos listados podremos interactuar filtrando los datos por medio de varios filtro, incluso ordenarlos alfabeticamente en orden creciente y decreciente si se desea.\\

[[ Imagen de un listado ejemplo con algunos filtros ]]

%++++++++++++++++++++++++++++++++++++++++++++++++++++++++++++++++++++++++++++++
\section{Busquedas por ID}
\label{3:sec5}

Tambien se pueden realizar busquedas directas por el ID del socio, en la que nos redirecciona a directamente  a la ficha t�cnica del socio.\\

[[ Imagen de busquedas por ID ]]

%++++++++++++++++++++++++++++++++++++++++++++++++++++++++++++++++++++++++++++++
\section{Exportaciones}
\label{3:sec6}

Los resultados de los listados los podemos exportar a Google Drive en formato de hoja de calculo, con solo darle al bot�n de exportar debajo del listado. Veamos un ejemplo con imagenes:\\

[[ Boton exportar]]\\

El resultado seria el siguiente:\\

[[ Documento en drive ]]

%++++++++++++++++++++++++++++++++++++++++++++++++++++++++++++++++++++++++++++++
\section{Importaciones}
\label{3:sec7}

GScout brinda a la organizaci�n de scout Aguere 70 la posibilidad de migrar su base de datos en \textbf{Access} a la propia aplicaci�n, siempre y cuando este fichero se transforme en un archivo .csv, con la herramienta MDB Tools por ejemplo,
y solo har�a falta subirla a la aplicaci�n, que lo dem�s lo hace la propia aplicaci�n de manera automatica. De esta manera tendremo los datos de la antigua base de datos en la nueva base de datos que pertenece a GScout.\\


[[ Imagen de subida de archivo csv ]]