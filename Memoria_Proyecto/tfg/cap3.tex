%%%%%%%%%%%%%%%%%%%%%%%%%%%%%%%%%%%%%%%%%%%%%%%%%%%%%%%%%%%%%%%%%%%%%%%%%%%%%%%
% Chapter 3: Descripción de la Aplicación
%%%%%%%%%%%%%%%%%%%%%%%%%%%%%%%%%%%%%%%%%%%%%%%%%%%%%%%%%%%%%%%%%%%%%%%%%%%%%%%

%++++++++++++++++++++++++++++++++++++++++++++++++++++++++++++++++++++++++++++++


En el cápitulo ~\ref{chapter:intro} se describió brevemente la aplicación. En este cápitulo nos expandiremos y eplicaremos toda la funcionalidad de GScout.

%++++++++++++++++++++++++++++++++++++++++++++++++++++++++++++++++++++++++++++++
\section{Usuarios}
\label{3:sec1}

La aplicación esta destinada para el uso de los empleados de la organización de scout Aguere 70, el inicio de sesión esta restringido, por tanto solo podrán 
acceder con su cuenta de Google perteneciente al dominio ``gruposcoutaguere70.com''. En principio todos los usuarios son genericos, pero la aplicación esta preparada
para clasificarlos según su cargo, y que cada usuario pueda desempeñar unas funciones u otras dependiendo del cargo asignado por el administrador.\\


\begin{figure}[H]
\begin{center}
\includegraphics[width=0.5\textwidth]{images/login.jpg}
\caption{Ventana de login}
\label{fig:ArbolBinario}
\end{center}
\end{figure}


%++++++++++++++++++++++++++++++++++++++++++++++++++++++++++++++++++++++++++++++
\section{Gestión de socios}
\label{3:sec2}

A continuación nombraremos las funcionalidades entorno a la gestión de socios.

\subsection{Creación de Socios}

Los usuarios podrán crear nuevos socios rellenando los formularios predefinidos en la aplicación, donde se clasifican los datos en personales, economicos, familiares y medicos.\\

\begin{figure}[H]
\begin{center}
\includegraphics[width=0.5\textwidth]{images/ejemplo_formulario_personal.jpg}
\caption{Ejemplo de Formulario}
\label{fig:ArbolBinario}
\end{center}
\end{figure}

\subsection{Visualización y Edición de Socios}

Una vez creados los socios podremos visualizar su información accediendo a la ficha de socio, donde la informacion se divide en 4 pestañas: Personales, Familia, Economicos y Médicos \\

\begin{figure}[H]
\begin{center}
\includegraphics[width=0.75\textwidth]{images/datos_personales.jpg}
\caption{Ficha del Socio}
\label{fig:ArbolBinario}
\end{center}
\end{figure}

Tambien se pueden editar sus datos, de una manera similar a la de creación de socios, simplemente buscamos el socio y le damos a editar, en la pestaña del tipo de información
que querramos cambiar.\\

\begin{figure}[H]
\begin{center}
\includegraphics[width=0.15\textwidth]{images/boton_editar.jpg}
\caption{Boton de Editar}
\label{fig:ArbolBinario}
\end{center}
\end{figure}

\subsection{Borrado de Socios}
Tambien podremos borrar los socios, de momento los socios se borran permanentemente, pero en el modelo de datos esta configurado pera poder implementar un metodo que en vez de borrarlos los socios pasen
a un estado inactivo.\\


\begin{figure}[H]
\begin{center}
\includegraphics[width=0.75\textwidth]{images/borrado_socios.jpg}
\caption{Borrado de Socios}
\label{fig:ArbolBinario}
\end{center}
\end{figure}

\subsection{Familiares}

Es obligado que cada socio pertenezca a una familia, la cual tendra un responsable que sera el indentificador de la familia, a la que se le pueden asignar varios socios, padres y/o tutores. En cierto modo, podremos
formar lo que seria un arbol de familia entre los socios, saber que socios conviven en la misma familia, saber quienes son los padres/tutores del socio, etc.\\

\begin{figure}[H]
\begin{center}
\includegraphics[width=0.75\textwidth]{images/familia_socio.jpg}
\caption{Borrado de Socios}
\label{fig:ArbolBinario}
\end{center}
\end{figure}

Tambien GScout nos da la opción de cambiar o crear una familia nueva, por cualquier circunstancia que pueda pasar, en la que se necesite modificar las familias ya existentes.\\

\begin{figure}[H]
\begin{center}
\includegraphics[width=0.75\textwidth]{images/cambio_familia.jpg}
\caption{Edición de Familia}
\label{fig:ArbolBinario}
\end{center}
\end{figure}


\subsection{Medicamentos}

Muchos de los socios podrían estar bajo medicación de algun tipo, por tanto existe un modulo en el que podemos anexar a los socios los datos de sus medicamentos, como el nombre, las pautas y las dosis correspondientes.\\

\begin{figure}[H]
\begin{center}
\includegraphics[width=0.75\textwidth]{images/medicamentos.jpg}
\caption{Medicamentos}
\label{fig:ArbolBinario}
\end{center}
\end{figure}

%++++++++++++++++++++++++++++++++++++++++++++++++++++++++++++++++++++++++++++++
\section{Cambios de unidad}
\label{3:sec3}

A petición de la organización scout Aguere 70 se creó una función que llamada \textbf{Cambio de Unidad} que consistes en analizar todos los socios, verificar su edad, y si procede cambiarlo a la sección/unidad acorde a su edad.\\
\begin{figure}[H]
\begin{center}
\begin{tabular}{c|p{25mm}c|p{25mm}|} \hline 
\textbf{Intervalo(Años)} & \textbf{Unidad} \\ \hline
0-11 &
Manada
\\
\hline

11-14 &
Tropa
\\
\hline

14-17 &
Esculta
\\
\hline

17-20 &
Rover
\\
\hline

20+ & 
Scouter
\\
\hline
\end{tabular}
\caption{Rango de edades}
\end{center}
\end{figure}

%++++++++++++++++++++++++++++++++++++++++++++++++++++++++++++++++++++++++++++++
\section{Listados de información}
\label{3:sec4}

Hay una sección en la aplicación en la que podemos listar los socios visualizando sus datos dependiendo del listado, es decir, existen un listado especifico para los datos personales y otro para los datos economicos.\\

En estos listados podremos interactuar filtrando los datos por medio de varios filtro, incluso ordenarlos alfabeticamente en orden creciente y decreciente si se desea.\\

\begin{figure}[H]
\begin{center}
\includegraphics[width=0.75\textwidth]{images/filtrado.jpg}
\caption{Aplicación de algunos filtros al listado}
\label{fig:ArbolBinario}
\end{center}
\end{figure}


%++++++++++++++++++++++++++++++++++++++++++++++++++++++++++++++++++++++++++++++
\section{Busquedas por ID}
\label{3:sec5}

Tambien se pueden realizar busquedas directas por el ID del socio, en la que nos redirecciona a directamente  a la ficha técnica del socio.\\

\begin{figure}[H]
\begin{center}
\includegraphics[width=0.40\textwidth]{images/busqueda_id.jpg}
\caption{Busqueda ID}
\label{fig:ArbolBinario}
\end{center}
\end{figure}

%++++++++++++++++++++++++++++++++++++++++++++++++++++++++++++++++++++++++++++++
\section{Exportaciones}
\label{3:sec6}

Los resultados de los listados los podemos exportar a Google Drive en formato de hoja de calculo, con solo darle al botón de exportar debajo del listado. Veamos un ejemplo con imagenes:\\

\begin{figure}[H]
\begin{center}
\includegraphics[width=0.75\textwidth]{images/listado_export.jpg}
\caption{Exportación de un listado}
\label{fig:ArbolBinario}
\end{center}
\end{figure}

El resultado seria el siguiente:\\

\begin{figure}[H]
\begin{center}
\includegraphics[width=0.75\textwidth]{images/result_export.jpg}
\caption{Resultado de la Esportación}
\label{fig:ArbolBinario}
\end{center}
\end{figure}

%++++++++++++++++++++++++++++++++++++++++++++++++++++++++++++++++++++++++++++++
\section{Importaciones}
\label{3:sec7}

GScout brinda a la organización de scout Aguere 70 la posibilidad de migrar su base de datos en \textbf{Access} a la propia aplicación, siempre y cuando este fichero se transforme en un archivo .csv, con la herramienta MDB Tools por ejemplo,
y solo haría falta subirla a la aplicación, que lo demás lo hace la propia aplicación de manera automatica. De esta manera tendremo los datos de la antigua base de datos en la nueva base de datos que pertenece a GScout.\\


\begin{figure}[H]
\begin{center}
\includegraphics[width=0.75\textwidth]{images/import_db.jpg}
\caption{Subiendo archivo csv para importar base de datos}
\label{fig:ArbolBinario}
\end{center}
\end{figure}