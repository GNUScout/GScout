%%%%%%%%%%%%%%%%%%%%%%%%%%%%%%%%%%%%%%%%%%%%%%%%%%%%%%%%%%%%%%%%%%%%%%%%%%%%%
% Chapter 6: Summary and Conlusions
%%%%%%%%%%%%%%%%%%%%%%%%%%%%%%%%%%%%%%%%%%%%%%%%%%%%%%%%%%%%%%%%%%%%%%%%%%%%%%%

%++++++++++++++++++++++++++++++++++++++++++++++++++++++++++++++++++++++++++++++

My experiences during the time of preparing this Master Thesis have been very successful. At the beginning
(before starting the project) I was afraid how to face it. I used to work in groups and with the help of
each other we can address any problem. Now I have to do it by myself. 
% the fact that distressed me do it myself, but it was not the case, with assistance from
With the assistance from my
tutor, and several hours of research and investigation looking for solutions to the proposed problems caused the project to go on forward.\\

On the other hand, the project I chose, developing an application for managing Scout groups, is very complete.
During the project development,  touched many contents that were taught in Computer Science degree. 
Regarding the project features, I want to mention this is a typical MVC application. As we used Django as the developing  
framework, our design was conducted in a MTV (model-temmplate-view) software architecture: a small variation of the MVC 
software architecture model.
We work with a tunned version of the framework,
Django-nonrel, to use non-relational database because the deployment is made on Google App Engine. This technology
only supports non-relational databases, based on the BigTable model. We also use integration
with Google APIs like Google+  for authentication and access to users personal data, and Google Drive to extend
application functionality, exporting data to spreadsheet documents in Drive. Regarding web design, we used
CSS for customizations, and code fragments in JavaScript and jQuery to create dynamism in the interface of the website,
while deploying some features on the client.\\

Another thing I found interesting is that the project is being developed to achieve the requirements and needs of a real Scout organization.
I have the oportunity to have several meetings with them, to collect requirements, show progress, adapt to the client,
refocus the project if necessary, etc. This me I bring a lot of experience when it comes to see more or less how are the cycles of
software development in a real environment.\\

As it has been commented in future work  chapter, the application development is focused on achieving the basic functionality 
concerning scout members management, while retaining a vision in order to gradually increase GScout functionality 
to meet the needs that originally Aguere 70 Scout organization gave us. For example, modules to incorporate the treasury, library, waiting lists, etc.\\

Therefore, we can say that GScout is an open application, fully prepared to add more functionality, with only the fact of annexing
modules to the application.

%---------------------------------------------------------------------------------

