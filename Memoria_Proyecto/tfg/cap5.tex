%%%%%%%%%%%%%%%%%%%%%%%%%%%%%%%%%%%%%%%%%%%%%%%%%%%%%%%%%%%%%%%%%%%%%%%%%%%%%
% Chapter 5: Conclusiones y Trabajos Futuros 
%%%%%%%%%%%%%%%%%%%%%%%%%%%%%%%%%%%%%%%%%%%%%%%%%%%%%%%%%%%%%%%%%%%%%%%%%%%%%%%

%++++++++++++++++++++++++++++++++++++++++++++++++++++++++++++++++++++++++++++++



Mis experiencias a la hora de elaborar el Trabajo de Fin de Grado han sido muy satisfactorias, a pesar de que al principio 
(antes de empezar con el proyecto) tenia miedo de como afrontarlo, ya que al estar acostumbrado a trabajar en grupo y ayudarnos 
mutuamente para salir del paso ante un problema, el hecho de hacerlo yo solo me angustiaba, pero no fue el caso, con ayuda del 
tutor, y varias horas de investigación y indagación buscando soluciones a los problemas propuesto hicieron que el proyecto saliera adelante.\\

Por otro lado el proyecto que elegí, el de crear una aplicación para la gestión de grupos Scout, me pareció muy completo, a lo 
largo del proyecto se tocan muchos contenidos que se ha impartido en la carrera de Grado en Informática. Dentro de lo que es la 
arquitectura de software, se utiliza el modelo de Vista-Controlador que no ofrece Django, pero trabajamos con una versión peculiar,
Django-nonrel, para poder utilizar base de datos no relacionales ya que el despliegue se efectuara con Google App Engine, y esta tecnología 
solo soporta base de datos no relacionales, de hecho tiene una propia, la BigTable de Google. A parte de esto se utilizaron también integración 
con APIs de Google, como Google + para autenticaciones y obtención de datos personales de los usuarios, y la API de Google Drive para extender 
las funciones de la aplicación y poder exportar datos a documentos de hoja de cálculos en Drive. Ademas de que en el diseño web se utilizan 
archivos CSS para las personalizaciones, fragmentos y códigos en Javascript y Jquery para crear dinamismo en la apariencia e interfaz de las paginas web.\\

Otra cosa que me pareció interesante, es que el proyecto se está elaborando para cumplir las expectativas y necesidades de una organización de 
scout real, por tanto teníamos que realizar varias reuniones con ellos, para recolectar requisitos, mostrar avances, adaptarlo al cliente, 
volver a enfocar el proyecto si fuese necesario, etc. Esto me aporto mucha experiencia a la hora de ver mas o menos como son los ciclos de
desarrollo de software en un entorno real.\\


En cuanto a trabajos futuros, como bien se ha comentado en los cápitulos anteriores, el desarrollo de la aplicación se enfocó en cubrir las funciones básicas
referentes a la getión de los socios scout, pero conservando una visión de futuro con el fin de al paso del tiempo ir ampliando las funcionalidades de GScout
para satisfacer las necesidades que en un principio la organización de scout Aguere 70 no dió, como por ejemplo incorporar un modulo para la tesorería, biblioteca, listas de espera, etc.\\

Por lo tanto podremos decir que GScoput es una aplicación abierta y está totalmente preparada para añadirle más funcionalidades, con solo el hecho de anexar
modulos a la aplicación.