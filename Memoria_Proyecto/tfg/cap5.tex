%%%%%%%%%%%%%%%%%%%%%%%%%%%%%%%%%%%%%%%%%%%%%%%%%%%%%%%%%%%%%%%%%%%%%%%%%%%%%
% Chapter 5: Conclusiones y Trabajos Futuros 
%%%%%%%%%%%%%%%%%%%%%%%%%%%%%%%%%%%%%%%%%%%%%%%%%%%%%%%%%%%%%%%%%%%%%%%%%%%%%%%

%++++++++++++++++++++++++++++++++++++++++++++++++++++++++++++++++++++++++++++++



Mis experiencias a la hora de elaborar el Trabajo de Fin de Grado han sido muy satisfactorias, a pesar de que al principio 
(antes de empezar con el proyecto) tenia miedo de como afrontarlo, ya que al estar acostumbrado a trabajar en grupo y ayudarnos 
mutuamente para salir del paso ante un problema, el hecho de hacerlo yo solo me angustiaba, pero no fue el caso, con ayuda del 
tutor, y varias horas de investigaci�n y indagaci�n buscando soluciones a los problemas propuesto hicieron que el proyecto saliera adelante.\\

Por otro lado el proyecto que eleg�, el de crear una aplicaci�n para la gesti�n de grupos Scout, me pareci� muy completo, a lo 
largo del proyecto se tocan muchos contenidos que se ha impartido en la carrera de Grado en Inform�tica. Dentro de lo que es la 
arquitectura de software, se utiliza el modelo de Vista-Controlador que no ofrece Django, pero trabajamos con una versi�n peculiar,
Django-nonrel, para poder utilizar base de datos no relacionales ya que el despliegue se efectuara con Google App Engine, y esta tecnolog�a 
solo soporta base de datos no relacionales, de hecho tiene una propia, la BigTable de Google. A parte de esto se utilizaron tambi�n integraci�n 
con APIs de Google, como Google + para autenticaciones y obtenci�n de datos personales de los usuarios, y la API de Google Drive para extender 
las funciones de la aplicaci�n y poder exportar datos a documentos de hoja de c�lculos en Drive. Ademas de que en el dise�o web se utilizan 
archivos CSS para las personalizaciones, fragmentos y c�digos en Javascript y Jquery para crear dinamismo en la apariencia e interfaz de las paginas web.\\

Otra cosa que me pareci� interesante, es que el proyecto se est� elaborando para cumplir las expectativas y necesidades de una organizaci�n de 
scout real, por tanto ten�amos que realizar varias reuniones con ellos, para recolectar requisitos, mostrar avances, adaptarlo al cliente, 
volver a enfocar el proyecto si fuese necesario, etc. Esto me aporto mucha experiencia a la hora de ver mas o menos como son los ciclos de
desarrollo de software en un entorno real.\\


En cuanto a trabajos futuros, como bien se ha comentado en los c�pitulos anteriores, el desarrollo de la aplicaci�n se enfoc� en cubrir las funciones b�sicas
referentes a la geti�n de los socios scout, pero conservando una visi�n de futuro con el fin de al paso del tiempo ir ampliando las funcionalidades de GScout
para satisfacer las necesidades que en un principio la organizaci�n de scout Aguere 70 no di�, como por ejemplo incorporar un modulo para la tesorer�a, biblioteca, listas de espera, etc.\\

Por lo tanto podremos decir que GScoput es una aplicaci�n abierta y est� totalmente preparada para a�adirle m�s funcionalidades, con solo el hecho de anexar
modulos a la aplicaci�n.