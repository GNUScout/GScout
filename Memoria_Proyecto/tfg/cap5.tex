%%%%%%%%%%%%%%%%%%%%%%%%%%%%%%%%%%%%%%%%%%%%%%%%%%%%%%%%%%%%%%%%%%%%%%%%%%%%%
% Chapter 5: Conclusiones y Trabajos Futuros 
%%%%%%%%%%%%%%%%%%%%%%%%%%%%%%%%%%%%%%%%%%%%%%%%%%%%%%%%%%%%%%%%%%%%%%%%%%%%%%%

%++++++++++++++++++++++++++++++++++++++++++++++++++++++++++++++++++++++++++++++



Mis experiencias a la hora de elaborar el Trabajo de Fin de Grado han sido muy satisfactorias, a pesar de que al principio 
(antes de empezar con el proyecto) tenía miedo de como afrontarlo. Al estar acostumbrado a trabajar en grupo y ayudarnos 
mutuamente para salir del paso ante un problema, el hecho de hacerlo yo sólo me angustiaba. No fue el caso, con ayuda del 
tutor, y varias horas de investigación e indagación buscando soluciones a los problemas propuestos hicieron que el proyecto saliera adelante.\\

Por otro lado el proyecto que elegí, el de crear una aplicación para la gestión de grupos Scout, me pareció muy completo, a lo 
largo del proyecto se tocan muchos contenidos que se han impartido en la carrera de Grado en Informática. Dentro de lo que es la 
arquitectura de software, se utiliza el modelo de Template-View que nos ofrece Django. Trabajamos con una versión peculiar,
Django-nonrel, para poder utilizar base de datos no relacionales ya que el despliegue se efectuará con Google App Engine, y esta tecnología 
solo soporta este tipo de base de datos, basadas en el modelo BigTable. A parte de esto se utilizaron también integración 
con APIs de Google, como Google+ para autenticacion y obtención de datos personales de los usuarios, y Google Drive para extender 
las funciones de la aplicación y poder exportar datos a documentos de hoja de cálculos en Drive. Además, en el diseño web se utilizan 
hojas de estilo CSS para las personalizaciones, fragmentos y códigos en Javascript y JQuery para crear dinamismo en la apariencia de interfaz de las páginas web, 
a la vez que descargan algunas funcionalidades en el cliente.\\

Otra cosa que me pareció interesante, es que el proyecto se está elaborando para cumplir las expectativas y necesidades de una organización 
Scout real, por tanto teníamos que realizar varias reuniones con ellos, para obtener requisitos, mostrar avances, adaptarlo al cliente, 
volver a enfocar el proyecto si fuese necesario, etc. Esto me aporto mucha experiencia a la hora de ver más o menos cómo son los ciclos de
desarrollo de software en un entorno real.\\


En cuanto a trabajos futuros, como bien se ha comentado en los cápitulos anteriores, el desarrollo de la aplicación se enfocó en cubrir las funciones básicas
referentes a la gestión de los socios scout, pero conservando una visión de futuro con el fin de ir ampliando las funcionalidades de GScout
para satisfacer las necesidades que en un principio la organización Scout Aguere 70 nos dió, como por ejemplo incorporar módulos para la tesorería, biblioteca, listas de espera, etc.\\

Por lo tanto podremos decir que GScout es una aplicación abierta y está totalmente preparada para añadirle más funcionalidades, con sólo el hecho de anexar
módulos a la aplicación.
