%%%%%%%%%%%%%%%%%%%%%%%%%%%%%%%%%%%%%%%%%%%%%%%%%%%%%%%%%%%%%%%%%%%%%%%%%%%%%%%
% Chapter 2: Título del capítulo 2
%%%%%%%%%%%%%%%%%%%%%%%%%%%%%%%%%%%%%%%%%%%%%%%%%%%%%%%%%%%%%%%%%%%%%%%%%%%%%%%

%++++++++++++++++++++++++++++++++++++++++++++++++++++++++++++++++++++++++++++++

En el capítulo anterior se ha introducido los antecedentes como el estado actual del proyecto, 
nombrado sus herramientas, actividades y periodos de desarrollo. Ahora nos vamos enfocar y describir lo que es el
entorno de desarrollo de la aplicación\\

Partimos de una implementación en PHP, que poseía las funcionalidades básicas, pero el problema era que la base de datos
no era compatile con nuestra tecnología, debido a que seguía el model entidad-relación, y nosotros al contrario, necesitábamos
crear una base de datos no relacional. Por otro lado teníamos de ejemplo la aplicación web ``online scout'', la cual era muy completa
pero era de pago, y nuestro objetivo era hacer una aplicación gratuita adaptada al cliente, en nuestro caso la organización de scout Aguere 70 de La Laguna.


%++++++++++++++++++++++++++++++++++++++++++++++++++++++++++++++++++++++++++++++

\section{Bases de la aplicación}
\label{2:sec1}

La aplicación GScout esta programada principalmente en lenguaje python, bajo el framework de django-nonrel,  
ya que como se ha comentado antes se necesitaba este framework especifico para trabajar con bases de datos no relacionales,
debido a que como usamos Google App Engine para el despliegue, un requerimiento que tiene esta tecnologia es que solo trabaja con
bases de datos no relacionales. En un principio se habia propuesto el uso de pivotaltracker para el seguimiento de la aplicación, 
pero como solo habia un desarrollador en proceso, pues se descartó la idea y se limito a tener un repositorio gestor de versiones por medio de
github, donde se van guardando los cambios oportunos, y segun la información de los ``commit'' se puede ver lo que se ha hecho.

\section{Modelo Vista Controlador: Django}
\label{2:sec2}

Django es un framework de desarrollo web de código abierto, escrito en Python, que cumple en cierta medida el paradigma del Modelo Vista Controlador. 
Fue desarrollado en origen para gestionar varias páginas orientadas a noticias de la World Company de Lawrence, Kansas, 
y fue liberada al público bajo una licencia BSD en julio de 2005; el framework fue nombrado en alusión al guitarrista de jazz gitano Django Reinhardt.

\section{Google App Engine}
\label{2:sec3}

Google App Engine te permite ejecutar tus aplicaciones web en la infraestructura de Google.Las 
aplicaciones App Engine son fáciles de crear, de mantener y de ampliar al ir aumentando el tráfico 
y las necesidades de almacenamiento de datos. Con App Engine no necesitarás utilizar ningún servidor: 
solo tendrás que subir tu aplicación para que los usuarios puedan empezar a utilizarla.\\

Puedes proporcionar a la aplicación tu propio nombre de dominio (como, por ejemplo, http://www.example.com/) 
a través de Google Apps. También puedes proporcionarle un nombre que esté disponible en el dominio appspot.com. 
Podrás compartir tu aplicación con todo el mundo o limitar el acceso a los miembros de tu organización.\\


Google App Engine admite aplicaciones escritas en varios lenguajes de programación.\\


Google App Engine permite desarrollar fácilmente aplicaciones que se ejecuten de forma fiable, 
incluso con pesadas cargas de trabajo y grandes cantidades de datos. App Engine incluye las siguientes funciones:

\begin{itemize}
  \item Servidor web dinámico, totalmente compatible con las tecnologías web más comunes,
  \item Almacenamiento permanente con funciones de consulta, clasificación y transacciones,
  \item Escalado automático y distribución de carga,
  \item API para autenticar usuarios y enviar correo electrónico a través de Google Accounts,
  \item Un completo entorno de desarrollo local que simula Google App Engine en tu equipo,
  \item Colas de tareas que realizan trabajos fuera del ámbito de una solicitud web,
  \item Tareas programadas para activar eventos en momentos determinados y en intervalos regulares.
\end{itemize}

\section{Google APIs}
\label{2:sec4}

En la aplicación se usaron dos APIs de Google para aumentar y mejorar su funcionalidad. 


\subsection{Google Plus}
Como GAE ya posee un modulo de autenticación por medio de Google Auth,
en el proyecto se modifico para que se hiciera por medio de Google Plus, generando unas credenciales que se guardaran en el usuarios
para que la aplicaccion pueda tener determinados permisos y poder obtener información que proporciona Google Plus. El objetivo de la incorporación de esta 
API a nuestra aplicación es evitar almacenar datos personales de los integrantes de la organización que usa la aplicación en una base de datos, sino usar directamente los
datos que nos proporciona Google Plus, como nombre, apellidos, foto de perfíl, dirección, etc.

\subsection{Google Drive}
Uno de los objetivos de la aplicación en cuanto a funcionalidad era poder exportar una tabla filtrada o no con los datos de los socios, como estamos utilizando tecnologia Google,
decidimos que la mejor forma era utilizar la API de Google Drive, para crear documentos en formato de hoja de calculo en el entorno de Google Drive. Para ello fue necesario
modificar las credenciales que se almacenaban en los usuarios, para darle permiso y poder utilizar las funciones que nos proporciona dicha API.

\section{GitHub}
\label{2:sec5}
GitHub es un software para alojar proyectos utilizando el sistema de control de versiones Git. 
El código se almacena de forma pública, aunque también se puede hacer de forma privada, creando una cuenta de pago.\\

Nuestra aplicación esta alojada en un repertorio público, cuyo repertorio se facilitara en la sección de enlances.

\section{JavaScript y JQuery}
\label{2:sec6}


\subsection{JavaScript}
JavaScript es un lenguaje de programación interpretado, orientado a objetos, basado en prototipos, imperativo, débilmente tipado y dinámico. 
Se utiliza principalmente en su forma del lado del cliente (client-side), implementado como parte de un navegador web permitiendo mejoras en 
la interfaz de usuario y páginas web dinámicas, en bases de datos locales al navegador, etc.

\subsection{JQuery}
jQuery es una biblioteca de JavaScript, creada inicialmente por John Resig, que permite simplificar la manera de interactuar con los documentos HTML,
manipular el árbol DOM, manejar eventos, desarrollar animaciones (FLV) y agregar interacción con la técnica AJAX a páginas web. Cuyas caracteristicas son:

\begin{itemize}
  \item Selección de elementos DOM.
  \item Interactividad y modificaciones del árbol DOM, incluyendo soporte para CSS 1-3 y un plugin básico de XPath.
  \item Eventos.
  \item Manipulación de la hoja de estilos CSS.
  \item Efectos y animaciones.
  \item Animaciones personalizadas.
  \item AJAX.
  \item Soporta extensiones.
  \item Utilidades varias como obtener información del navegador, operar con objetos y vectores, funciones para rutinas comunes, etc.
  \item Compatible con los navegadores Mozilla Firefox 2.0+, Internet Explorer 6+, Safari 3+, Opera 10.6+ y Google Chrome 8+.
\end{itemize}

\section{CSS y Bootstrap}
\label{2:sec7}

Para generar el estilo de nuestra aplicación utilizamos las tecnologías que se descriran a continuación.

\subsection{CSS (Hojas de estilo)}
Las hojas de estilo en cascada (Cascading Style Sheets, o sus siglas CSS) hacen referencia a un 
lenguaje de hojas de estilos usado para describir la presentación semántica (el aspecto y formato) de un 
documento escrito en lenguaje de marcas. Su aplicación más común es dar estilo a páginas webs escritas en lenguaje HTML y XHTML, 
pero también puede ser aplicado a cualquier tipo de documentos XML, incluyendo SVG y XUL.

\subsection{Bootstrap}
Para agilizar el maquetado de las páginas web de la aplicación utilizamos Twitter Bootstrap, que es una colección de herramientas de software
libre para la creación de sitios y aplicaciones web. Contiene plantillas de diseño basadas en HTML y CSS con tipografías, 
formularios, botones, gráficos, barras de navegación y demás componentes de interfaz, así como extensiones opcionales de JavaScript.

\section{Pylint}
\label{2:sec8}
Pylint es una herramienta que básicamente tiene como función analizar código Python en busca de errores o síntomas de mala calidad en el código fuente. Con el uso de esta herramienta
hacemos que nuestro código tenga mayor calidad y sea mas legible por lo demás.

